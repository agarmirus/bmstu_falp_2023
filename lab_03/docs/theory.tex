\chapter{Теоретические вопросы}

\section{Базис Lisp}

Базис~--- минимальный набор конструкций языка, на основе которого могут быть построены вычисления.

Базис языка составляют:
\begin{itemize}
	\item атомы;
	\item структуры;
	\item базовые функции;
	\item базовые функционалы.
\end{itemize}

\section{Классификация функций}

Функции в Lisp делятся на три класса:
\begin{itemize}
	\item базисные,
	\item ядра,
	\item пользовательские.
\end{itemize}

Также, функции Lisp делятся на чистые (обычные, математические) и формы~--- функции, которые особым образом обрабатывают свои аргументы.

Базисные функции, в свою очередь, классифицируются следующим образом:
\begin{itemize}
	\item селекторы (car, cdr и т.д.),
	\item конструкторы (cons),
	\item предикаты (atom),
	\item сравнения (eq).
\end{itemize}

\section{Способы создания функций}

В Lisp можно определять безымянные функции и функции с именем.

Безымянная функция:
\begin{lstlisting}
	(lambda (x1 x2 ... xn) f)
\end{lstlisting}

Функция с именем:
\begin{lstlisting}
	(defun <name> (x1 x2 ... xn) f)
\end{lstlisting}

Вызов:
\begin{lstlisting}
	(<name> a1 a2 ... an)
	((lambda (x1 x2 ... xn) f) a1 a2 ... an)
\end{lstlisting}

\section{Работа функций cond, if, and, or}

Вычисляет выражения test до тех пор, пока одно из них не окажется истинным.
Если с ним не связаны какие-либо выражения, возвращается значение его самого, в противном случае связанные с ним выражения вычисляются по очереди и возвращается значение последнего. Если ни одно тестовое выражение не оказалось истинным, возвращается Nil.
\begin{lstlisting}
(cond (test1 value1) (test2 value2) ... (testN valueN))
\end{lstlisting}

Функция if является сокращением cond.
Если test равен T, то возвращается значение выражения valT.
В ином случае возвращается значение выражения valNil, если оно передано функции if в качестве аргумента, иначе возвращается Nil.
\begin{lstlisting}
(if test valT [valNil])
\end{lstlisting}

Функция and вычисляет выражения друг за другом до тех пор, пока одно из них не будет ложным.
В таком случае возвращается Nil, в противном случае~--- значение последнего аргумента.
Если выражения не заданы, возвращается T.
\begin{lstlisting}
(and ard1 arg2 ... argN)
\end{lstlisting}

Функция or вычисляет выражения друг за другом до тех пор, пока одно из них не будет истиным.
В таком случае возвращается само значение, в противном случае~--- Nil.
Если выражения не заданы, возвращается Nil.
\begin{lstlisting}
(or ard1 arg2 ... argN)
\end{lstlisting}

\clearpage
