\chapter{Теоретические вопросы}

% В лиспе функции носят частичный характер (могут полность или неполностью отрабатываться на аргументах).
% Функции не всегда корректно работают при различных аргументах. Поэтому необходимо понимать, как работать функция.

% Куча: что это, что еще есть кроме кучи, как устроена озу с физической точки зрения и как это вообще работает

% Что происходит при включении компьютера

% Что такое car, cdr?

% Как работают cons и list?

\section{Элементы языка: определение, синтаксис, представление в памяти}

Вся информация (данные и программы) в языке Lisp представляются в виде символьных выражений~--- S-выражений. По определению:

\begin{center}
	S-выражение ::= <атом> | <точечная пара>
\end{center}

В языке Lisp допустимы синтаксические конструкции, в которых используются элементарные или более сложные конструкции.

Атомы~--- элементарные конструкции, которые являются:
\begin{itemize}
	\item символами (идентификаторами)~--- синтаксически, это набор литер (букв и цифр), начинающихся с буквы;
	\item специальными символами {T,~Nil} (используются для обозначения логических констант);
	\item самоопределимыми атомами~--- натуральные, дробные и вещественные числа, строки (последовательности символов, заключенных в двойные апострофы);
\end{itemize}

Более сложные данные~--- списки и точечные пары, которые строятся из унифицированных структур~--- блоков памяти.
Синтаксис списков и точечных пар выглядит следующим образом:

\textbf{Точечные пары ::= (<атом>.<атом>) | (<атом>.<точечная пара>) |}

\textbf{(<точечная пара>.<атом>) | (<точечная пара>.<точечная пара>);}

\textbf{Список ::= <пустой список> | <непустой список>,}

\textbf{<пустой список> ::= ( ) | Nil,}

\textbf{<непустой список>::= (<первый элемент> . <хвост>),}

\textbf{<первый элемент> ::= <S-выражение>,}

\textbf{<хвост> ::= <список>.}

Любая непустая структура Lisp в памяти представляется бинарным узлом, хранящей два указателя: на голову (первый элемент) и хвост (все остальное).

\section{Особенности языка Lisp. Структура программы. Символ апостроф}

Lisp является языком символьной обработки.
Из этого следует, что система не различает программы и данные.

Основная конструкция, которая используется в Lisp,~--- это списки. Список~--- это структура, которая может быть пустой или непустой; если непустой, то он имеет хотя бы один элемент, называемый головой, а остальные~--- хвостом.

Основой языка Lisp является лямбда-исчисление, суть которого заключается в том, что любые вычисления могут быть преобразованы в композицию функций.

Безымянная функция:
\begin{lstlisting}
	(lambda (x1 x2 ... xn) f)
\end{lstlisting}

Функция с именем:
\begin{lstlisting}
	(defun <name> (x1 x2 ... xn) f)
\end{lstlisting}

Вызов:
\begin{lstlisting}
	(<name> a1 a2 ... an)
	((lambda (x1 x2 ... xn) f) a1 a2 ... an)
\end{lstlisting}

По умолчанию система воспринимает конструкции как программы, при этом голова списка интерпретируется как имя функции, а остальные элементы~--- как ее фактические параметры. Для обозначения данных используется блокировка вычислений с помощью quote или символа апострофа:
\begin{lstlisting}
	(quote (A B C))
	`(A B C)
\end{lstlisting}

\section{Базис языка Lisp. Ядро языка}

Базис~--- минимальный набор конструкций языка, на основе которого могут быть построены вычисления.

Базис языка составляют:
\begin{itemize}
	\item атомы;
	\item структуры;
	\item базовые функции;
	\item базовые функционалы.
\end{itemize}

Базовые функции языка:
\begin{itemize}
	\item quote (блокировка вычисления);
	\item eval (принудительное вычисление);
	\item car (разыменование указателя на голову);
	\item cdr (разыменование указателя на хвост);
	\item cons (создание бинарного узла);
	\item atom (проверяет, является ли аргумент атомом или нет);
	\item cond (условная функция);
	\item eq (проверка на равенство, применима только к символьным атомам).
\end{itemize}

\clearpage
