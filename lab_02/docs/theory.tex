\chapter{Теоретические вопросы}

\section{Базис Lisp}

Базис~--- минимальный набор конструкций языка, на основе которого могут быть построены вычисления.

Базис языка составляют:
\begin{itemize}
	\item атомы;
	\item структуры;
	\item базовые функции;
	\item базовые функционалы.
\end{itemize}

\section{Классификация функций}

Функции в Lisp делятся на три класса:
\begin{itemize}
	\item базисные,
	\item ядра,
	\item пользовательские.
\end{itemize}

Базисные функции, в свою очередь, классифицируются следующим образом:
\begin{itemize}
	\item селекторы (car, cdr и т.д.),
	\item конструкторы (cons),
	\item предикаты (atom),
	\item сравнения (eq).
\end{itemize}

\section{Способы создания функций}

В Lisp можно определять безымянные функции и функции с именем.

Безымянная функция:
\begin{lstlisting}
	(lambda (x1 x2 ... xn) f)
\end{lstlisting}

Функция с именем:
\begin{lstlisting}
	(defun <name> (x1 x2 ... xn) f)
\end{lstlisting}

Вызов:
\begin{lstlisting}
	(<name> a1 a2 ... an)
	((lambda (x1 x2 ... xn) f) a1 a2 ... an)
\end{lstlisting}

\section{Функции car, cdr, eq, eql, equal, equalp}

Функция car разыменовывает первый указатель бинарного узла:
\begin{lstlisting}
(car '(1 2 3)) 		-> 1
(car '((1 2) 3)) 	-> (1 2)
(car '(1 . 2)) 		-> 1
(car '(1))				-> 1
\end{lstlisting}

Функция cdr разыменовывает второй указатель бинарного узла:
\begin{lstlisting}
(cdr '(1 2 3)) 		-> (2 3)
(cdr '((1 2) 3)) 	-> (3)
(cdr '(1 . 2)) 		-> 2
(car '(1))				-> Nil
\end{lstlisting}

Функция eq осуществляет сравнение только символьных атомов:
\begin{lstlisting}
(eq 'a 'a)				-> T
(eq 'a 'b)				-> Nil
(eq 1 1)					-> T
(eq 2 1)					-> Nil
(eq 2 2.0)				-> Nil
(eq 2.00 2.0)			-> Nil
\end{lstlisting}

Функция eql осуществляет сравнение между символьных атомов или чисел:
\begin{lstlisting}
(eql 'a 'a)				-> T
(eql 'a 'b)				-> Nil
(eql 1 1)					-> T
(eql 2 1)					-> Nil
(eql 2 2.0)				-> Nil
(eql 2.00 2.0)		-> T
\end{lstlisting}

Функция equal осуществляет сравнение символьных атомов, чисел или списков:
\begin{lstlisting}
(equal 'a 'a)								-> T
(equal 'a 'b)								-> Nil
(equal 1 1)									-> T
(equal 2 1)									-> Nil
(equal 2 2.0)								-> Nil
(equal 2.00 2.0)						-> T
(equal '(1 2 3) '(1 2 3))		-> T
(equal '(1 2 3) '(1 2 4))		-> Nil
\end{lstlisting}

Функция equalp осуществляет сравнение символьных атомов, чисел разных видов или списков:
\begin{lstlisting}
(equalp 'a 'a)								-> T
(equalp 'a 'b)								-> Nil
(equalp 1 1)									-> T
(equalp 2 1)									-> Nil
(equalp 2 2.0)								-> T
(equalp 2.00 2.0)							-> T
(equalp '(1 2 3) '(1 2 3))		-> T
(equalp '(1 2 3) '(1 2 4))		-> Nil
\end{lstlisting}

\section{Назначение и отличие в работе функций cons и list}

Функция cons создает бинарный узел, первый указатель которого ссылается на значение первого переданного аргумента, а второй~--- на значение второго.

Функция list создает список, состоящий из переданных функции аргументов.
Если функции не были переданы фактические параметры, то создается пустой список.

Основные отличия в работе функций cons и list:
\begin{itemize}
	\item функция cons принимает фиксированное количество аргументов, функция list~--- произвольное;
	\item функция cons создает один бинарный узел, функция list~--- список.
\end{itemize}

\clearpage
